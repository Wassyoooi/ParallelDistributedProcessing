\documentclass[14pt, oneside]{article}     	% use "amsart" instead of "article" for AMSLaTeX format
\usepackage{geometry}                		% See geometry.pdf to learn the layout options. There are lots.

\geometry{left=25mm,right=25mm,top=30mm,bottom=30mm}
\geometry{letterpaper}                   		% ... or a4paper or a5paper or ...
%\geometry{landscape}                		% Activate for rotated page geometry
%\usepackage[parfill]{parskip}    		% Activate to begin paragraphs with an empty line rather than an indent
\usepackage[dvipdfmx]{graphicx}				% Use pdf, png, jpg, or eps§ with pdflatex; use eps in DVI mode
								% TeX will automatically convert eps --> pdf in pdflatex
\usepackage{amssymb}

\usepackage{listings}
\renewcommand{\lstlistingname}{コード}
%SetFonts
\lstset{%
  language={vhdl},
  basicstyle={\small},%
  identifierstyle={\small},%
  commentstyle={\small\itshape},%
  keywordstyle={\small\bfseries},%
  ndkeywordstyle={\small},%
  stringstyle={\small\ttfamily},
  frame={single},
  breaklines=true,
  columns=[l]{fullflexible},%
  numbers=left,%
  xrightmargin=0zw,%
  xleftmargin=3zw,%
  numberstyle={\scriptsize},%
  stepnumber=1,
  numbersep=1zw,%
  lineskip=-0.5ex%
}
%SetFonts


\title{並列分散処理  最終レポート}
\author{大城 慶知}
%\date{}							% Activate to display a given date or no date

\begin{document}
\maketitle

\section{テーマ}
Pythonにおける非同期処理を用いたI/Oの並列処理を行う。
\section{Python並列処理の基礎知識}

\subsection{スレッドの制約}
Pythonでは、GIL(Global Interpreter Lock)と呼ばれる制約がある。
GILとは、Pythonを実行した際に一つだけしかスレッドのリソースを起動できない制約である。
そのため、PythonのCPUバウンドの並列処理はプロセスを使って、I/Oバウンドの並列処理はスレッドを行う事が多い。

\begin{figure}[h]
  \centering
  \includegraphics[width=10cm]{multithred_iobound.png}
  \caption{林檎の図}
\end{figure}

\begin{figure}[h]
  \centering
  \includegraphics[width=10cm]{multithred_cpubound.png}
  \caption{林檎の図}
\end{figure}

\subsection{プロセスを用いた}


\begin{lstlisting}[caption=シンプレクス法プログラム]

\end{lstlisting}


\section{動作確認}
\vspace{5mm}

動作確認は求められている動作を考慮しながら行った。
SW1・SW2・SW3とボタンがあり、それぞれ以下の表のように動作が対応している。
\begin{center}
  \begin{table}[htb]
    \begin{tabular}{|l|c|} \hline
      ボタン名 & 動作 \\ \hline \hline
      SW1 & スタート・ストップ \\ \hline
      SW2 & リセット \\ \hline
      SW3 & LAPボタン \\ \hline
    \end{tabular}
  \end{table}
\end{center}

まず、前提としてそれぞれが動作することを確認する。

SW1は、、LCDモジュールにLEDと同様のタイマー機能が表示され、スタートとストップの動きが正常であるか。
SW2は、リセットボタンを押すとLCDモジュールの表示が初期の状態に戻るか。
SW3は、押したタイミングでLCDモジュールにタイマー表示が適切にされるかどうか。

\vspace{5mm}
次に、それぞれの動作が重なったときに適切な動作ができるかを確認する。

ストップウォッチが止まっているときに、SW3ボタンを無効にできているか。
リセットボタンを押した際に、LAPボタンで変更した表示も初期化できているか。




\section{感想・意見}

\section{数理計画・解答}
3.次の線形計画問題を解きなさい。絵画存在する場合は最適解(x_1, X_2)とそのときのZを答えなさい。(20点)

max z = x_1-x_2
subject to
-2x1 + x2 <= 2
x_1 - 2x_2 <= 2
x_1 + x_2 <= 5
x1, x2 >= 0

用語の理解
概念

最適化の古典的問題
シンプレクス法
停止条件存在条件
応用問題
0-1計画問題の解法

\begin{thebibliography}{9}
  \bibitem{harris} Pythonをとりまく並行/非同期の話,  https://tell-k.github.io/pyconjp2017/
  \bibitem{susan} S. M. Smith and J. M. Brady,
    ``SUSAN|A new approach to low level image processing,'' Int. J. Comput.
    Vis., vol.23, no.1, pp.45-78, May 1997.
\end{thebibliography}


\end{document}
