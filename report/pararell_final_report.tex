\documentclass[14pt, oneside]{article}     	% use "amsart" instead of "article" for AMSLaTeX format
\usepackage{geometry}                		% See geometry.pdf to learn the layout options. There are lots.
\usepackage[dvipdfmx]{hyperref}

\geometry{left=25mm,right=25mm,top=30mm,bottom=30mm}
\geometry{letterpaper}                   		% ... or a4paper or a5paper or ...
%\geometry{landscape}                		% Activate for rotated page geometry
%\usepackage[parfill]{parskip}    		% Activate to begin paragraphs with an empty line rather than an indent
\usepackage[dvipdfmx]{graphicx}				% Use pdf, png, jpg, or eps§ with pdflatex; use eps in DVI mode
								% TeX will automatically convert eps --> pdf in pdflatex
\usepackage{amssymb}

\usepackage{listings}
\renewcommand{\lstlistingname}{コード}
%SetFonts
\lstset{%
  language={vhdl},
  basicstyle={\small},%
  identifierstyle={\small},%
  commentstyle={\small\itshape},%
  keywordstyle={\small\bfseries},%
  ndkeywordstyle={\small},%
  stringstyle={\small\ttfamily},
  frame={single},
  breaklines=true,
  columns=[l]{fullflexible},%
  numbers=left,%
  xrightmargin=0zw,%
  xleftmargin=3zw,%
  numberstyle={\scriptsize},%
  stepnumber=1,
  numbersep=1zw,%
  lineskip=-0.5ex%
}
%SetFonts

\title{並列分散処理  最終レポート}
\author{チームE 大城 慶知 眞榮城 隆守 伊波卓浩 宮良友也}
%\date{}							% Activate to display a given date or no date

\begin{document}
\maketitle

\section*{最終報告書に載せるやつ(あとで消すやつ)}

演習の背景、目的、方法、結果、考察をA410ページ以内で適切にまとめる。
個々のメンバーの役割分担を明記する。記載がない場合、あるいは、曖昧な場合には、
減点の対象となる。例えば、あるタスクを複数名で担当した場合でも、個々のメンバーの
役割をできる限り区別して説明する。
最終報告書にはソースコードのgithubリポジトリも記載する。

\section{テーマ}
Pythonにおける非同期処理を用いたI/Oの並列処理を行う。

\section{Python並列処理の基礎知識}

\subsection{スレッドの制約}
Pythonでは、GIL(Global Interpreter Lock)と呼ばれる制約がある。
GILとは、Pythonを実行した際に一つだけしかスレッドのリソースを起動できない制約である。
そのため、PythonのCPUバウンドの並列処理はプロセスを使って、I/Oバウンドの並列処理はスレッドを行う事が多い。

\begin{figure}[h]
  \centering
  \includegraphics[width=10cm]{multithred_iobound.png}
  \caption{林檎の図}
\end{figure}

\begin{figure}[h]
  \centering
  \includegraphics[width=10cm]{multithred_cpubound.png}
  \caption{林檎の図}
\end{figure}

\subsection{プロセスを用いた}

\begin{lstlisting}[caption=シンプレクス法プログラム]

\end{lstlisting}

\section{実験方法}
HTTPのGETを用いて実験を行った.
GETを複数回実行する場合を考えると,
逐次処理の場合ではレスポンスがあるまで
次のGETを送信することができない.
これを並列処理により
レスポンスを待つことなく
次のGETを実行した.
これにより効率よく
GETを実行し,
結果を受け取ることができると
想定した.

\section{実行結果}

example1.pyは逐次処理をしてくれる
スクリプト。\\
example2.pyはコルーチンを
20個一気に呼び出す。\\
example3.pyはコルーチンを
5個ずつ4回呼び出している。\\
example4.pyはスレッドを使用。\\


実行結果を以下に示す。

・example1.py
\begin{lstlisting}
% python example1.py
1.7710762023925781
3.6481258869171143
5.234184980392456
6.819124937057495
8.661988735198975
10.357062816619873
12.127899885177612
13.939480066299438
15.66226577758789
17.136775016784668
18.71082091331482
22.602702856063843
24.2241370677948
25.92159104347229
27.67835807800293
29.555225133895874
31.347226858139038
33.13314199447632
34.98287010192871
36.86141490936279
\end{lstlisting}


・example2.py
\begin{lstlisting}
% python example2.py 
6.204233169555664
6.204301834106445
6.204308032989502
6.204311847686768
6.204314947128296
6.204318046569824
6.204322099685669
6.204324960708618
6.2043280601501465
6.204332113265991
6.20433497428894
6.204339027404785
6.2043421268463135
6.204345941543579
6.204349040985107
6.204353094100952
6.204355955123901
6.204360008239746
6.204363107681274
6.204365968704224
\end{lstlisting}

・example3.py
\begin{lstlisting}
% python example3.py
1.983794927597046
1.9838616847991943
1.983867883682251
1.9838709831237793
1.983874797821045
3.5528876781463623
3.552910804748535
3.5529158115386963
3.5529186725616455
3.552921772003174
5.265967845916748
5.265992879867554
5.265997648239136
5.2660017013549805
5.266004800796509
6.929059743881226
6.929084777832031
6.929089784622192
6.929093837738037
6.929096937179565
\end{lstlisting}

・example4.py
\begin{lstlisting}
あいうえお
\end{lstlisting}

\section{考察}






\section{感想・意見}
\section*{GitHubのURL}

\subsection*{https://github.com/e165719/ParallelDistributedProcessing}

\end{document}
