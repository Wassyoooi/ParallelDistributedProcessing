\documentclass[14pt, oneside]{article}     	% use "amsart" instead of "article" for AMSLaTeX format
\usepackage{geometry}                		% See geometry.pdf to learn the layout options. There are lots.
\usepackage[dvipdfmx]{hyperref}

\geometry{left=25mm,right=25mm,top=30mm,bottom=30mm}
\geometry{letterpaper}                   		% ... or a4paper or a5paper or ...
%\geometry{landscape}                		% Activate for rotated page geometry
%\usepackage[parfill]{parskip}    		% Activate to begin paragraphs with an empty line rather than an indent
\usepackage[dvipdfmx]{graphicx}				% Use pdf, png, jpg, or eps§ with pdflatex; use eps in DVI mode
								% TeX will automatically convert eps --> pdf in pdflatex
\usepackage{amssymb}

\usepackage{listings}
\renewcommand{\lstlistingname}{コード}
%SetFonts
\lstset{%
  language={vhdl},
  basicstyle={\small},%
  identifierstyle={\small},%
  commentstyle={\small\itshape},%
  keywordstyle={\small\bfseries},%
  ndkeywordstyle={\small},%
  stringstyle={\small\ttfamily},
  frame={single},
  breaklines=true,
  columns=[l]{fullflexible},%
  numbers=left,%
  xrightmargin=0zw,%
  xleftmargin=3zw,%
  numberstyle={\scriptsize},%
  stepnumber=1,
  numbersep=1zw,%
  lineskip=-0.5ex%
}
%SetFonts

\title{並列分散処理  最終レポート}
\author{チームE 大城 慶知 眞榮城 隆守 伊波卓浩 宮良友也}
%\date{}							% Activate to display a given date or no date

\begin{document}
\maketitle

\section*{最終報告書に載せるやつ(あとで消すやつ)}

演習の背景、目的、方法、結果、考察をA410ページ以内で適切にまとめる。
個々のメンバーの役割分担を明記する。記載がない場合、あるいは、曖昧な場合には、
減点の対象となる。例えば、あるタスクを複数名で担当した場合でも、個々のメンバーの
役割をできる限り区別して説明する。
最終報告書にはソースコードのgithubリポジトリも記載する。

\section{テーマ}
Pythonにおける非同期処理を用いたI/Oの並列処理を行う。

\section{Python並列処理の基礎知識}

\subsection{スレッドの制約}
Pythonでは、GIL(Global Interpreter Lock)と呼ばれる制約がある。
GILとは、Pythonを実行した際に一つだけしかスレッドのリソースを起動できない制約である。
そのため、PythonのCPUバウンドの並列処理はプロセスを使って、I/Oバウンドの並列処理はスレッドを行う事が多い。

\begin{figure}[h]
  \centering
  \includegraphics[width=10cm]{multithred_iobound.png}
  \caption{林檎の図}
\end{figure}

\begin{figure}[h]
  \centering
  \includegraphics[width=10cm]{multithred_cpubound.png}
  \caption{林檎の図}
\end{figure}

\subsection{プロセスを用いた}

\begin{lstlisting}[caption=シンプレクス法プログラム]

\end{lstlisting}

\section{実験方法}
とうっとぅるー

\section{実行結果}

こうなりました。

しゅごいぃぃぃぃぃぃぃぃぃぃぃぃぃっっ


\section{考察}






\section{感想・意見}
\section*{GitHubのURL}

\subsection*{https://github.com/e165719/ParallelDistributedProcessing}

\end{document}
