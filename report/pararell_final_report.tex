\documentclass[14pt, oneside]{article}     	% use "amsart" instead of "article" for AMSLaTeX format
\usepackage{geometry}                		% See geometry.pdf to learn the layout options. There are lots.
\usepackage[dvipdfmx]{hyperref}
\usepackage{here}

\geometry{left=25mm,right=25mm,top=30mm,bottom=30mm}
\geometry{letterpaper}                   		% ... or a4paper or a5paper or ...
%\geometry{landscape}                		% Activate for rotated page geometry
%\usepackage[parfill]{parskip}    		% Activate to begin paragraphs with an empty line rather than an indent
\usepackage[dvipdfmx]{graphicx}				% Use pdf, png, jpg, or eps§ with pdflatex; use eps in DVI mode
								% TeX will automatically convert eps --> pdf in pdflatex
\usepackage{amssymb}

\usepackage{listings}
\renewcommand{\lstlistingname}{コード}
%SetFonts
\lstset{%
  language={Python},
  basicstyle={\small},%
  identifierstyle={\small},%
  commentstyle={\small\itshape},%
  keywordstyle={\small\bfseries},%
  ndkeywordstyle={\small},%
  stringstyle={\small\ttfamily},
  frame={single},
  breaklines=true,
  columns=[l]{fullflexible},%
  numbers=left,%
  xrightmargin=0zw,%
  xleftmargin=3zw,%
  numberstyle={\scriptsize},%
  stepnumber=1,
  numbersep=1zw,%
  lineskip=-0.5ex%
}
%SetFonts

\title{並列分散処理  最終レポート}
\author{チームE 大城 慶知 眞榮城 隆守 伊波卓浩 宮良友也}
%\date{}							% Activate to display a given date or no date

\begin{document}
\maketitle

% \section*{最終報告書に載せるやつ(あとで消すやつ)}
%
% 演習の背景、目的、方法、結果、考察をA410ページ以内で適切にまとめる。
% 個々のメンバーの役割分担を明記する。記載がない場合、あるいは、曖昧な場合には、
% 減点の対象となる。例えば、あるタスクを複数名で担当した場合でも、個々のメンバーの
% 役割をできる限り区別して説明する。
% 最終報告書にはソースコードのgithubリポジトリも記載する。

\section{演習の背景}
「講義で説明した並列分散処理を実践し、他者に有益な情報となるように共有せよ。」という課題が渡された。

b3前期はメンバーが忙しく時間も取れないので軽量かつ有益な情報をということで、
GPUマシンを使った並列処理を断腸の思いで断念し、Pythonにおける非同期処理を用いたI/Oの並列処理を行うことにした。

\section{目的}
PythonのネットワークI/Oに対して、async awaitの使うことで非同期処理を行い、
スレッドとconcurrentを用いて並列化を行うことで速度向上率を調べる。

\section{メンバーの担当}

大城 慶知 総括

眞榮城 隆守・宮良友也 Pythonファイルの記述。グラフの生成

伊波卓浩 FizzBuzzのデモ作成

\section{Pythonの事前知識}

\subsection{スレッドの制約}
Pythonでは、GIL(Global Interpreter Lock)と呼ばれる制約がある。
GILとは、Figure \ref{multithred_iobound}と Figure\ref{multithred_cpubound}のように、
CPUバウンドでPythonを実行した際に一つだけしかスレッドのリソースを起動できない制約である。
そのため、PythonのCPUバウンドの並列処理はプロセスを使って、I/Oバウンドの並列処理はスレッドを行う事が多い。

\begin{figure}[H]
  \centering
  \includegraphics[width=12cm]{multithred_iobound.png}
  \caption{マルチスレッドI/Oバウンド}
  \label{multithred_iobound}
\end{figure}

\begin{figure}[H]
  \centering
  \includegraphics[width=12cm]{multithred_cpubound.png}
  \caption{マルチスレッドCPUバウンド}
  \label{multithred_cpubound}
\end{figure}

\newpage

\subsection{asyncとawait}
asyncとawaitがどのような処理を行うのか以下のコード \ref{FizzBuzz} を実行して動作を確認した。
async_awaitはasyncで定義した関数をawaitで呼び出し実行することで、コルーチンが生成されプログラム実行状態に移り、
コルーチンごとに非同期処理を行うコードとなっている。

\lstinputlisting[caption=非同期処理のFizzBuzz,label=FizzBuzz]{../FizzBuzz/FizzBuzz.py}

\begin{lstlisting}[caption=FizzBuzz実行結果]
9:12:Fizz
4:12:Fizz
5:12:Fizz
2:12:Fizz
6:12:Fizz
3:12:Fizz
7:12:Fizz
1:12:Fizz
8:12:Fizz
9:13:13
4:13:13
5:13:13
2:13:13
6:13:13
3:13:13
7:13:13
1:13:13
8:13:13
9:14:14
4:14:14
5:14:14
2:14:14
6:14:14
3:14:14
7:14:14
1:14:14
8:14:14
9:15:FizzBuzz
4:15:FizzBuzz
\end{lstlisting}

コード \ref{FizzBuzz}は、30までFizzBuzzを10回行う処理を非同期処理で実装している。

実行結果からFizzBuzz関数を呼び出す際にsleepしていると、
コンテキストスイッチと呼ばれる、実行できるプログラムを先に実行する処理を行うからだと考えられる。
ただ実行はCPUバウンドなので1スレッドしか利用できずに、速度としてはかなり遅かった。

% \newpage

\section{実験方法}

HTTPのGETを用いて実験を行う。
GETを複数回実行する場合を考えると、
逐次処理の場合ではレスポンスがあるまで
次のGETを送信することができない。
async-awaitまたはThreadPoolExecutorを用いることで、
レスポンスを待つことなく
次のGETを実行することで効率よく
結果を受け取ることができると
想定した。

それぞれ逐次処理、async-await(20回渡し)、async-awaitのみ(5回ずつ)、async-await + ThreadPoolExecutorに分けて
琉球大学情報工学科HPへのGETを20回連続で行うコードを記述した。

また、上記の順でコード \ref{example1}、コード \ref{example2}、コード \ref{example3}、コード \ref{example4}に示したとおりである。

\lstinputlisting[caption=逐次処理,label=example1]{../HttpRequest/example1.py}


\lstinputlisting[caption=async-awaitのみ(20個渡し),label=example2]{../HttpRequest/example2.py}


\lstinputlisting[caption=async-awaitのみ(4回ずつ),label=example3]{../HttpRequest/example3.py}

\lstinputlisting[caption=async-await + ThreadPoolExecutor,label=example4]{../HttpRequest/example4.py}

% \newpage

\section{実行結果}

% example1.pyは逐次処理をしてくれる
% スクリプト。\\
% example2.pyはコルーチンを
% 20個一気に呼び出す。\\
% example3.pyはコルーチンを
% 5個ずつ4回呼び出している。\\
% example4.pyはスレッドを使用。\\


実行結果をFigure\ref{result}に示した。
Figure \ref{result}を見ると、逐次処理、async-awaitのみ(4回ずつ)、
async-awaitのみ(20個渡し)、async-await + ThreadPoolExecutorの順で時間がかかっていることがわかる。

% \lstinputlisting[caption=逐次処理実行結果.py,label=1]{../HttpRequest/sequentially}
%
% \lstinputlisting[caption=async_await20,label=2]{../HttpRequest/async_await20}
%
% \lstinputlisting[caption=async_await5,label=3]{../HttpRequest/async_await5}
%
% \lstinputlisting[caption=async_await・threadあり,label=4]{../HttpRequest/thread}

\begin{figure}[H]
  \centering
  \includegraphics[width=13cm]{time2.png}
  \caption{実行結果}
  \label{result}
\end{figure}


\section{考察}
async-awaitを用いた非同期処理は、コルーチンを呼び出して行っているので高速であることがグラフから読み取れた。
また、async-awaitでプログラムを4回に分けて渡したとしても実行時間が変わらなかったことから、
async-awaitは動的にタスクを増やしても実行時間に影響が出にくいことが考えられる。

async-await + ThreadPoolExecutorは少しだけ実行速度が早くなったことから、
Threadの数が増えて並列度が増したと考えられる。

つまり、async-awaitを用いてプログラムの並行に走らせ、Threadで並列に実行させることで
より並列度が高い処理を実装することが今回の実行結果から読み取れる。
しかし、Thread数を2とすると実行速度が下がったことからThreadの生成コストが並列度悪くする場合もあるので、
使う際にはThread数に気をつける必要がある。

\section{感想・意見}
b3後期が忙しかったために凝った内容にできなかった。
PythonでのCPUバウンドでの並行処理はあまり使われているイメージが無く、あったとしてもライブラリで完結している
ものが多いのでCPUバウンドでPythonを取り上げるのがふさわしいのかわからないと感じた。

しかし、GPUを用いた機械学習や画像処理はPythonでもだいぶ重要になってくるので率先してやりたかったが、
期間内に0から学ぶことを考えると手が出しづらかった。
\section{GitHubのURL}
\subsection*{https://github.com/e165719/ParallelDistributedProcessing}
\begin{thebibliography}{9}
  \bibitem{harris} Pythonをとりまく並行/非同期の話,  https://tell-k.github.io/pyconjp2017/
\end{thebibliography}

\end{document}
